\cleardoublepage
\chapter{Conclusioni}
  \section{Obiettivi raggiunti}
    \subsection{Funzionalità componenti sviluppate}
    Al termine del periodo di stage sono stati raggiunti tutti i requisiti obbligatori e parte di quelli opzionali.\\
    In particolare:
    \begin{itemize}
      \item è stata ridefinita l'architettura server, gestendo in modo opportuno il canale di comunicazione e gli utenti lato amministratore;
      \item è stato reimplementato il sistema di permessi in modo da concedere a più utenti di accedere alla conversazione;
      \item è stata integrata nell'architettura l'interfaccia grafica, in modo da fornire una visione delle funzionalità disponibili.
    \end{itemize}
    \subsection{Sviluppi futuri}
      Durante lo sviluppo del progetto sono nate nuove idee riguardo alcune funzionalità da implementare.\\
      Una di queste è di fornire degli strumenti che manipolino lo streaming video in modo da permettere il tracciamento e l'evidenziazione di alcune parti di un macchinario che necessitano di assistenza.\\
      Questo permetterebbe di mostrare in modo più preciso uno step di un procedimento più o meno complicato, rendendo l'assistenza più efficiente e proficua.\\\\
      Un'altra opzione è quella di unire l'applicazione Android con il sistema sviluppato che, a causa del poco tempo rimasto, non è stato possibile fare.\\
      Bisognerà capire quanto difficile possa risultare la modifica dell'applicazione Android per integrare i tipi e i metodi che sono stati definiti per il sistema che è stato sviluppato.
  \section{Obiettivi personali}
    \subsection{Risultati ottenuti}
      Uno degli obiettivi che mi ero fissato prima di cominciare lo stage era di imparare qualcosa di nuovo, che mi portasse ad ampliare le mie conoscenze in un ambito che sta prendendo piede e che mi interessa.\\
      Questo obiettivo è stato raggiunto perché, anche se ho utilizzato una tecnologia già conosciuta, mi sono reso conto di non conoscerla così bene e di quanti altri modi ci fossero per poterla utilizzare.\\
      Inoltre il campo della realtà virtuale mi sta appassionando molto e, confrontandomi con persone che si intendono di più dell'argomento, mi permette di conoscere opinioni e idee che sono diverse dalla mia.
    \subsection{Valutazione personale}
      Inizialmente è stato difficile perché essendo la prima esperienza lavorativa di questo calibro, non ero completamente preparato a questa mole di lavoro.\\
      Tuttavia mi sono dovuto subito mettere in gioco, per studiare, analizzare le componenti già sviluppate e reimplementarle e alla fine sono soddisfatto del livello che ha raggiunto il progetto.\\
      Sono anche soddisfatto di non aver avuto problemi ad integrarmi nel team, dove ho trovato persone competenti e molto disponibili.\\\\
      Infine posso dire di essere orgoglioso di aver lavorato in un'azienda che si sta sviluppando molto e, anche se ha incontrato diversi problemi in passato, non ha mai perso l'entusiasmo e la voglia di mettersi in gioco.
\vfill
\newpage
\null
\thispagestyle{empty}
