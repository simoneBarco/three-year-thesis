\addcontentsline{toc}{chapter}{Glossario}
\markboth{Glossario}{Glossario}
\chapter*{Glossario}
    \subsubsection{\label{API}API}
      Acronimo di Application Programming Interface. Indica un insieme di procedure disponibili al programmatore, raggruppate a formare un set di strumenti specifici per espletare un determinato compito in un programma. La finalità è raggiungere il riuso del codice, tramite l'ottenimento di un'astrazione a più alto livello.
    \subsubsection{\label{CSS}CSS}
      Acronimo di Cascading Style Sheets. Linguaggio usato per la formattazione di documenti HTML. La sua introduzione si è resa necessaria per permettere una programmazione più chiara e semplice da utilizzare, tramite la separazione dei contenuti dalla loro formattazione.
    \subsubsection{\label{Groovy}Groovy}
      Linguaggio di programmazione ad oggetti per la piattaforma Java, alternativo al linguaggio Java. La sua sintassi è molto simile a quella di Java, infatti permette di interagire in modo trasparente con altro codice Java.
    \subsubsection{\label{HTML}HTML}
      Acronimo di HyperText Markup Language. È un linguaggio di markup che ha come scopo quello di gestire i contenuti associandone o specificandone allo stesso tempo il layout, per realizzare una pagina web, tramite l'utilizzo di tag diversi.
    \subsubsection{\label{HTTP}HTTP}
      Acronimo di HyperText Transfert Protocolo. È un protocollo a livello applicativo usato come sistema principale per la trasmissione di informazione nel web.
    \subsubsection{\label{IDE}IDE}
      Acronimo di Integrated Development Environment. È un software che aiuta i programmatori nella scrittura del codice sorgente. Fornisce strumenti e funzionalità di supporto alla fase di sviluppo e debugging, oltre che a segnalare errori di sintassi del codice durante la fase di scrittura.
    \subsubsection{\label{IoT}IoT}
      Acronimo di Internet of Things. Si riferisce all'estensione di Internet ad oggetti comuni, in modo che diventino più intelligenti. Questo porta a comunicare dati su sé stessi e sul mondo che li circonda e allo stesso tempo di accedere a informazioni nella rete.
    \subsubsection{\label{Java}Java}
      È un linguaggio di programmazione ad alto livello, orientato agli oggetti e a tipizzazione statica, progettato per essere il più possibile indipendente dalla piattaforma di esecuzione.
    \subsubsection{\label{JS}JavaScript}
      È un linguaggio di scripting orientato agli oggetti e agli eventi. Viene utilizzato nella programmazione web lato client per la creazione di effetti che scatenano funzioni invocate da eventi, a loro volta innescati dall'utente. Viene utilizzato anche per creare app per più sistemi operativi.
    \subsubsection{\label{JSON}JSON}
      Formato adatto allo scambio di dati tra applicazioni client-server. Si basta su un sottoinsieme di JavaScript ma ne è indipendente.
    \subsubsection{\label{JVM}JVM}
      Acronimo di Java Virtual Machine. È la componente della piattaforma Java che esegue i programmi che sono stati tradotti in bytecode dopo una compilazione.
    \subsubsection{\label{PMI}PMI}
      Acronimo di Piccole e Medie Imprese.
    \subsubsection{\label{PM}Project Manager}
      È il responsabile dell'organizzazione dei processi e della loro pianificazione all'interno di un progetto.
    \subsubsection{\label{RA}Reactive Application}
      È un tipo di programma che segue i principi descritti nel Reactive Manifesto, che dà le indicazioni per seguire il paradigma di programmazione Reactive.
    \subsubsection{\label{RMI}RMI}
      Acronimo di Remote Method Invocation. È una tecnologia che, utilizzata nel contesto del linguaggio di programmazione Java, consente a processi Java distribuiti di comunicare attraverso una rete.
    \subsubsection{\label{Ruby}Ruby}
      È un linguaggio di programmazione completamente ad oggetti. Si differenzia dal C++ per essere molto più dinamico, infatti è possibile aggiungere o modificare metodi a runtime. Un'altra particolarità è che il suo tipo non è definito dalla classe che lo istanzia, ma dall'insieme dei metodi che possiede.
    \subsubsection{\label{Scala}Scala}
      È un linguaggio di programmazione di tipo general-purpose multiparadigma che integra le caratteristiche sia dei linguaggi orientati agli oggetti che dei linguaggi funzionali.
    \subsubsection{\label{Sprint}sprint}
      Periodo temporale in cui devono essere portati a termine dei task.
    \subsubsection{\label{Task}task}
      È un compito da portare a termine. Richiede poche ore di lavoro e può essere preso in carico da una sola persona.
    \subsubsection{\label{Tool}toolkit}
      È un insieme di strumenti software di base per facilitare e uniformare lo sviluppo di applicazioni più complesse.
    \subsubsection{\label{TCP}TCP}
      Acronimo di Transmission Controll Protocol. È un protocollo di rete a pacchetto di livello trasporto che si occupa di controllo trasmissione. Questo protocollo rende affidabile la comunicazione dati in rete tra mittente e destinatario.
    \subsubsection{\label{Wear}wearable}
      È un dispositivo elettronico indossabile o impiantabile. In generale offrono funzionalità di notifica legate agli smartphone oppure presentano dei sensori. Questo tipo di dispositivi è un esempio di dispositivo IoT.
    \subsubsection{\label{Web}WebSocket}
      È una tecnologia web che fornisce canali di comunicazione full-duplex attraverso una singola connessione TCP.
      \newpage
      \null
      \thispagestyle{empty}
      
